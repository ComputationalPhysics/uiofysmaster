\definecolor{listingsstringcolor}{rgb}{0,0.5,0}
\definecolor{listingskeywordcolor}{rgb}{0.5,0.5,0.0}
\definecolor{listingsbasiccolor}{rgb}{0.0,0.0,0.0}
% \definecolor{listingskeywordcolor}{rgb}{0.0,0.0,0.7}
\definecolor{listingsnumbercolor}{rgb}{0.0,0.0,1.0}
\definecolor{listingscommentcolor}{rgb}{0.4,0.4,0.4}
\definecolor{listingsbackgroundcolor}{rgb}{0.975,0.975,0.975}
\definecolor{listingsrulecolor}{rgb}{0.86,0.86,0.86}
\definecolor{listingsidentifiercolor}{rgb}{0.0,0.0,0.0}
\definecolor{listingsclasscolor}{rgb}{0.5,0.0,0.5}
\definecolor{listingsmembercolor}{rgb}{0.5,0.0,0.0}
\definecolor{listingsdirectivecolor}{rgb}{0.0,0.0,0.5}
% \definecolor{listingsvariablecolor}{rgb}{0.5,0.0,0.5}

\RequirePackage{iftex}

\newcommand{\listingsfont}{\sffamily}
\lstset {
    language=c++,
    numbers=none,
    breaklines=true,
    tabsize=2,
    backgroundcolor=\color{listingsbackgroundcolor},
    breakatwhitespace=true,         % sets if automatic breaks should only happen at whitespace
    breaklines=true,                 % sets automatic line breaking
  %   numbers=left,                    % where to put the line-numbers; possible values are (none, left, right)
    numbersep=5pt,                   % how far the line-numbers are from the code
    frame=lrtb,                    % adds a frame around the code
    framexleftmargin=7pt,
    framexrightmargin=7pt,
    framextopmargin=7pt,
    framexbottommargin=7pt,
    xleftmargin=18pt,
    xrightmargin=18pt,
    rulecolor=\color{listingsrulecolor},
    tabsize=2,                       % sets default tabsize to 2 spaces
    literate={-}{{\textendash}}1 {å}{{\aa}}1 {æ}{{\ae}}1 {ø}{{\oslash}}1,
    showstringspaces=false,
    captionpos=b,
    basicstyle=\color{listingsbasiccolor}\footnotesize\listingsfont,
    keywordstyle=\color{listingskeywordcolor}\footnotesize\listingsfont,
    directivestyle=\color{listingsdirectivecolor}\footnotesize\listingsfont,
    stringstyle=\color{listingsstringcolor}\footnotesize\listingsfont,
    commentstyle=\color{listingscommentcolor}\footnotesize\listingsfont,
    numberstyle=\color{listingsnumbercolor}\footnotesize\listingsfont,
    identifierstyle=\color{listingsidentifiercolor}\footnotesize\listingsfont,
    keywordstyle=[2]{\color{listingsclasscolor}\footnotesize\listingsfont},
    keywordstyle=[3]{\color{listingsmembercolor}\footnotesize\listingsfont},
    keywordstyle=[4]{\color{listingsdirectivecolor}\footnotesize\listingsfont},
}
\lstdefinelanguage{qmake}
{
    sensitive=true,
    morekeywords=[1]{
        absolute_path,
        basename,
        cat,
        clean_path,
        dirname,
        enumerate_vars,
        escape_expand,
        find,
        first,
        format_number,
        fromfile,
        getenv,
        join,
        last,
        list,
        lower,
        member,
        prompt,
        quote,
        re_escape,
        relative_path,
        replace,
        sprintf,
        resolve_depends,
        reverse,
        section,
        shadowed,
        shell_path,
        shell_quote,
        size,
        sort_depends,
        split,
        system,
        system_path,
        system_quote,
        unique,
        upper,
        val_escape,cache,
        contains,
        count,
        debug,
        defined,
        equals,
        error,
        eval,
        exists,
        export,
        files,
        for,
        greaterThan,
        if,
        include,
        infile,
        isActiveConfig,
        isEmpty,
        isEqual,
        lessThan,
        load,
        log,
        message,
        mkpath,
        requires,
        system,
        touch,
        unset,
        warning,
        write_file,
        packagesExist,
        prepareRecursiveTarget,
        qtCompileTest,
        qtHaveModule
    },
    morekeywords=[2]{
            CONFIG,
            DEFINES,
            DEF_FILE,
            DEPENDPATH,
            DEPLOYMENT,
            DEPLOYMENT_PLUGIN,
            DESTDIR,
            DISTFILES,
            DLLDESTDIR,
            FORMS,
            GUID,
            HEADERS,
            ICON,
            INCLUDEPATH,
            INSTALLS,
            LEXIMPLS,
            LEXOBJECTS,
            LEXSOURCES,
            LIBS,
            LITERAL_HASH,
            MAKEFILE,
            MAKEFILE_GENERATOR,
            MOC_DIR,
            OBJECTS,
            OBJECTS_DIR,
            POST_TARGETDEPS,
            PRE_TARGETDEPS,
            PRECOMPILED_HEADER,
            PWD,
            OUT_PWD,
            QMAKE,
            QMAKESPEC,
            QMAKE_AR_CMD,
            QMAKE_BUNDLE_DATA,
            QMAKE_BUNDLE_EXTENSION,
            QMAKE_CC,
            QMAKE_CFLAGS_DEBUG,
            QMAKE_CFLAGS_RELEASE,
            QMAKE_CFLAGS_SHLIB,
            QMAKE_CFLAGS_THREAD,
            QMAKE_CFLAGS_WARN_OFF,
            QMAKE_CFLAGS_WARN_ON,
            QMAKE_CLEAN,
            QMAKE_CXX,
            QMAKE_CXXFLAGS,
            QMAKE_CXXFLAGS_DEBUG,
            QMAKE_CXXFLAGS_RELEASE,
            QMAKE_CXXFLAGS_SHLIB,
            QMAKE_CXXFLAGS_THREAD,
            QMAKE_CXXFLAGS_WARN_OFF,
            QMAKE_CXXFLAGS_WARN_ON,
            QMAKE_DISTCLEAN,
            QMAKE_EXTENSION_SHLIB,
            QMAKE_EXT_MOC,
            QMAKE_EXT_UI,
            QMAKE_EXT_PRL,
            QMAKE_EXT_LEX,
            QMAKE_EXT_YACC,
            QMAKE_EXT_OBJ,
            QMAKE_EXT_CPP,
            QMAKE_EXT_H,
            QMAKE_EXTRA_COMPILERS,
            QMAKE_EXTRA_TARGETS,
            QMAKE_FAILED_REQUIREMENTS,
            QMAKE_FRAMEWORK_BUNDLE_NAME,
            QMAKE_FRAMEWORK_VERSION,
            QMAKE_INCDIR,
            QMAKE_INCDIR_EGL,
            QMAKE_INCDIR_OPENGL,
            QMAKE_INCDIR_OPENGL_ES2,
            QMAKE_INCDIR_OPENVG,
            QMAKE_INCDIR_X11,
            QMAKE_INFO_PLIST,
            QMAKE_LFLAGS,
            QMAKE_LFLAGS_CONSOLE,
            QMAKE_LFLAGS_DEBUG,
            QMAKE_LFLAGS_PLUGIN,
            QMAKE_LFLAGS_RPATH,
            QMAKE_LFLAGS_RPATHLINK,
            QMAKE_LFLAGS_RELEASE,
            QMAKE_LFLAGS_APP,
            QMAKE_LFLAGS_SHLIB,
            QMAKE_LFLAGS_SONAME,
            QMAKE_LFLAGS_THREAD,
            QMAKE_LFLAGS_WINDOWS,
            QMAKE_LIBDIR,
            QMAKE_LIBDIR_FLAGS,
            QMAKE_LIBDIR_EGL,
            QMAKE_LIBDIR_OPENGL,
            QMAKE_LIBDIR_OPENVG,
            QMAKE_LIBDIR_X11,
            QMAKE_LIBS,
            QMAKE_LIBS_EGL,
            QMAKE_LIBS_OPENGL,
            QMAKE_LIBS_OPENGL_ES1, QMAKE_LIBS_OPENGL_ES2,
            QMAKE_LIBS_OPENVG,
            QMAKE_LIBS_THREAD,
            QMAKE_LIBS_X11,
            QMAKE_LIB_FLAG,
            QMAKE_LINK_SHLIB_CMD,
            QMAKE_LN_SHLIB,
            QMAKE_POST_LINK,
            QMAKE_PRE_LINK,
            QMAKE_PROJECT_NAME,
            QMAKE_MAC_SDK,
            QMAKE_MACOSX_DEPLOYMENT_TARGET,
            QMAKE_MAKEFILE,
            QMAKE_QMAKE,
            QMAKE_RESOURCE_FLAGS,
            QMAKE_RPATHDIR,
            QMAKE_RPATHLINKDIR,
            QMAKE_RUN_CC,
            QMAKE_RUN_CC_IMP,
            QMAKE_RUN_CXX,
            QMAKE_RUN_CXX_IMP,
            QMAKE_TARGET,
            QT,
            QTPLUGIN,
            QT_VERSION,
            QT_MAJOR_VERSION,
            QT_MINOR_VERSION,
            QT_PATCH_VERSION,
            RC_FILE,
            RC_INCLUDEPATH,
            RCC_DIR,
            REQUIRES,
            RESOURCES,
            RES_FILE,
            SIGNATURE_FILE,
            SOURCES,
            SUBDIRS,
            TARGET,
            TARGET_EXT,
            TARGET_x,
            TARGET_x.y.z,
            TEMPLATE,
            TRANSLATIONS,
            UI_DIR,
            VERSION,
            VER_MAJ,
            VER_MIN,
            VER_PAT,
            VPATH,
            WINRT_MANIFEST,
            YACCSOURCES,
            _PRO_FILE_,
            _PRO_FILE_PWD_
    },
}
\input{package-config/listings-glsl.tex}
\lstdefinelanguage[std]{c++}[ISO]{c++}
{
    morekeywords=[2]{
        array,deque,forward_list,list,map,queue,set,stack,unordered_map,unordered_set,vector,
        unordered_multimap,multimap,multiset,
        basic_istringstream,basic_ostringstream,basic_stringstream,basic_stringbuf,
        istringstream,ostringstream,stringstream,stringbuf,wstringstream,wostringstream,wstringstream,wstringbuf,
        fstream,iomanip,ios,iosfwd,istream,ostream,sstream,streambuf,
        basic_ios,fpos,ios_base,io_errc,streamoff,streampos,streamsize,wstreampos,
        promise,packaged_task,future,shared_future,future_error,future_errc,future_status,launch,
        basic_ifstream,basic_ofstream,basic_fstream,basic_filebuf,
        ifstream,ofstream,fstream,filebuf,wifstream,wofstream,wfstream,wfilebuf,
        atomic,condition_variable,future,mutex,thread,
        algorithm,
        bitset,
        chrono,
        codecvt,
        complex,
        exception,
        functional,
        initializer_list,
        iterator,
        limits,
        locale,
        memory,
        new,
        numeric,
        random,
        ratio,
        regex,smatch,
        stdexcept,
        string,
        system_error,
        tuple,
        typeindex,
        typeinfo,
        type_traits,
        utility,
        valarray,
    },
    morecomment=[l][keywordstyle4]{\#include},
}


\lstdefinelanguage[armadillo]{c++}[std]{c++}
{
    morekeywords=[2]{
        Mat,mat,cx_mat,Col,colvec,vec,Row,rowvec,Cube,cube,field,SpMat,sp_mat,sp_cx_mat
    },
}


\input{package-config/listings-qt.tex}
\lstdefinelanguage{qml}
{
    keywords={typeof, new, true, false, catch, function, return, null, catch, switch, var, if, in, while, do, else, 
    case, break, import},
    keywords=[2]{class, export, boolean, throw, implements, import, this},
    sensitive=true,
    comment=[l]{//},
    morecomment=[s]{/*}{*/},
    morestring=[b]',
    morestring=[b]",
}


% \lstset{
%     literate={0}{{\textcolor{listingsnumbercolor}{0}}}{1}%
%              {1}{{\textcolor{listingsnumbercolor}{1}}}{1}%
%              {2}{{\textcolor{listingsnumbercolor}{2}}}{1}%
%              {3}{{\textcolor{listingsnumbercolor}{3}}}{1}%
%              {4}{{\textcolor{listingsnumbercolor}{4}}}{1}%
%              {5}{{\textcolor{listingsnumbercolor}{5}}}{1}%
%              {6}{{\textcolor{listingsnumbercolor}{6}}}{1}%
%              {7}{{\textcolor{listingsnumbercolor}{7}}}{1}%
%              {8}{{\textcolor{listingsnumbercolor}{8}}}{1}%
%              {9}{{\textcolor{listingsnumbercolor}{9}}}{1}%
%              {.0}{{\textcolor{listingsnumbercolor}{.0}}}{2}% Following is to ensure that only periods
%              {.1}{{\textcolor{listingsnumbercolor}{.1}}}{2}% followed by a digit are changed.
%              {.2}{{\textcolor{listingsnumbercolor}{.2}}}{2}%
%              {.3}{{\textcolor{listingsnumbercolor}{.3}}}{2}%
%              {.4}{{\textcolor{listingsnumbercolor}{.4}}}{2}%
%              {.5}{{\textcolor{listingsnumbercolor}{.5}}}{2}%
%              {.6}{{\textcolor{listingsnumbercolor}{.6}}}{2}%
%              {.7}{{\textcolor{listingsnumbercolor}{.7}}}{2}%
%              {.8}{{\textcolor{listingsnumbercolor}{.8}}}{2}%
%              {.9}{{\textcolor{listingsnumbercolor}{.9}}}{2}%
% }

\RequirePackage{calc}

%Define a reference depth. 
%You can choose either relative or absolute.
%--------------------------
\newlength{\DepthReference}
% \settodepth{\DepthReference}{g}%relative to a depth of a letter.
\setlength{\DepthReference}{0pt}%absolute value.

%Define a reference Height. 
%You can choose either relative or absolute.
%--------------------------
\newlength{\HeightReference}
% \settoheight{\HeightReference}{T}
\setlength{\HeightReference}{7pt}

\newlength{\Width}%
\newcommand{\lstinlinebox}[1]{%
  \settowidth{\Width}{\lstinline|#1|}%
  \fcolorbox{listingsrulecolor}{listingsbackgroundcolor}{%
    \raisebox{-\DepthReference}%
    {%
          \parbox[b][\HeightReference+\DepthReference][c]{\Width}{\centering \lstinline|#1|}%
    }%
  }%
}%